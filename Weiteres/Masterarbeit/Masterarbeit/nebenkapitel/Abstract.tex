%\section*{Zusammenfassung}
%\label{sec:zusammenfassung}

\section*{Abstract}
\label{sec:abstract}

Bei der Verwendung eines Conversational User Interface (CUI) bedient der Nutzer ein System über geschriebene und gesprochene Sprache. Anders als bei einem Graphical User Interface (GUI) muss die Antwort auf eine Frage nicht mehr gesucht, sondern kann direkt erfragt werden. Nachdem das System auf natürliche Art und Weise mit dem Nutzer interagiert, versteht es idealerweise alle Variationen einer Anfrage. Unabhängig davon, wie diese im Einzelnen formuliert sind. \\
Im Rahmen dieser Arbeit liegt der Fokus auf der Entwicklung eines Chatbots, also eines textbasierten CUIs. Die Arbeit umfasst die Konzeption und Implementierung eines solchen Systems im Kontext einer Raumbuchung. Dazu werden zunächst die Anforderungen an ein CUI im Allgemeinen und anschließend in Bezug auf ein Raumbuchungssystem betrachtet. In diesem Zusammenhang werden Methoden und Techniken aus dem Bereich Usability Engineering angewendet. Die Idee ist hierbei, multimodal mit dem CUI zu interagieren. So können GUI-Elemente wie beispielsweise Buttons oder Slider in die Konversation mit einfließen. Der Nutzer soll dabei jederzeit die Möglichkeit haben, auch textuelle Nachrichten an den Chatbot zu senden. Diese Ansätze werden mit Hilfe eines prototypischen Systems umgesetzt, überprüft und evaluiert. Durch die Umsetzung einiger der prototypisch erprobten Ansätze ist der Nutzer schließlich in der Lage, mit Hilfe eines Chatbots auf Basis einer Android Applikation nach freien Räumen zu suchen und einen Besprechungsraum zu reservieren.